\documentclass[11pt,a4paper]{IEEEtran}

\usepackage{epstopdf}
\usepackage[frenchb]{babel}
\usepackage[T1]{fontenc}
\usepackage{amsmath,amssymb,amsthm,bbm} % math, mathbb, mathbbm
\usepackage{url} %% pour citer les url par \url
\usepackage{listings}
\usepackage{subcaption}
\usepackage{tikz,pgfplots,pgf}
\usetikzlibrary{shapes.geometric}
\usepackage{subcaption}
\newtheorem{propriete}{Propriété}
\newtheorem{proposition}{Proposition}
\newtheorem{remarque}{Remarque}
\newtheorem*{exemple}{Exemple}
\newtheorem{definition}{Definition}

\author{charles\thanks{Ce cours est protégé par une licence CC BY-NC-SA. Contact: vincent.choqueuse@gmail.com}}
\definecolor{vert}{rgb}{0.2,0.6,0.4} 
\definecolor{rouge}{rgb}{0.9,0.,0.} 

\makeindex

\title{Introduction au Filtrage Numérique}
\date{\today}
\begin{document}
\markboth{Introduction au Filtrage Numérique (\today)}{}
\maketitle

\tikzstyle{start} = [minimum height=1cm]
\tikzstyle{box} = [rectangle, minimum width=1.25cm ,text width=1.9cm,minimum height=1.7cm,text centered, draw=black]
\tikzstyle{fgain} = [isosceles triangle,isosceles triangle apex angle=60,draw=black]
\tikzstyle{fbox} = [rectangle, minimum width=0.75cm,minimum height=0.75cm,text centered, draw=black]
\tikzstyle{foperation} = [circle, radius=0.2, minimum size=0.75cm,inner sep=0pt, draw=black]
\tikzstyle{stop} = [minimum height=1cm]


\begin{abstract}
Ce document est un cours d'introduction au filtrage numérique destiné à un public de Master ayant des bases en traitement du signal. Globalement, ce document a deux objectifs. Dans un premier temps, il présente les outils mathématiques utilisés pour analyser le comportement des filtres numériques tels que la transformée en $\mathcal{Z}$ et la transformée de Fourier. Puis dans un second temps, il décrit plusieurs techniques permettant de synthétiser un filtre numérique à partir d'un ensemble de spécifications. Le lecteur souhaitant aller plus loin dans ces deux thématiques pourra se réorienter vers la bible du filtrage numérique écrite par A. Oppenheim et R. Schafer \cite{opp} ou bien vers l'ouvrage de J. Smith \cite{smi}.
\end{abstract}

\begin{keywords}
Filtres numériques (FIR et IIR), Transformée en $\mathcal{Z}$, Transformée de Fourier, Invariance Impulsionnelle, Transformée Bilinéaire, Synthèse par fenêtrage, Méthode de Prony.
\end{keywords}

\section{Introduction}



Un filtre permet de "sculpter" un signal en supprimant ou en accentuant certaines composantes fréquentielles. Alors que les filtres analogiques reposent sur la conception de circuits électroniques, les filtres numériques eux sont souvent plus simples à mettre en oeuvre car ils reposent sur des algorithmes pouvant être codés dans n'importe quel langage de programmation (même si le C reste la norme pour les applications embarquées). Dans ce document, nous allons nous intéresser à la synthèse de filtres numériques à une entrée et une sortie. Ces filtres sont couramment utilisés dans un large panel d'applications. 

\subsection{Modélisation}

Un filtre numérique peut être modélisé par une fonction mathématique prenant en entrée un signal numérique et renvoyant en sortie un signal numérique (voir figure \ref{deffiltre}). Mathématiquement, un signal numérique est représenté par une suite de nombres où le $n^{ieme}$ nombre de la suite est noté $x[n]$ ($n \in \mathbb{Z}$). L'annexe \ref{app_sig} présente l'allure de plusieurs signaux numériques en fonction de $n$ (impulsion, echelon, exponentielle réelle, signal sinusoidal).

\begin{figure}[!h]
\centering
\begin{tikzpicture}[scale=0.5,node distance=2.5cm]
\node[start] (start) {$x[n]$};
\node[box,right of=start] (filter) { Filtre $T\{.\}$};
\node[stop,right of=filter] (end) {$y[n]$};
\draw[->,>=latex] (start) -- (filter);
\draw[->,>=latex] (filter) -- (end);
\node[below of=start,node distance=1cm]{Entrée};
\node[below of=end,node distance=1cm ]{Sortie};
\end{tikzpicture}
\caption{Filtre Numérique}
\label{deffiltre}
\end{figure}

Dans ce document, nous allons nous intéresser à la classe des filtres dits linéaires et invariant dans le temps (LTI). 

\begin{definition}[Filtre LTI] Un filtre LTI respecte les propriétés suivantes:
\begin{itemize}
\item Linéarité: Si l'entrée du filtre est égale à $x[n]=\alpha x_1[n]+\beta x_2[n]$, alors la sortie sera égale à $y[n]=\alpha y_1[n]+\beta y_2[n]$.
\item Invariance dans le temps: Si l'entrée du filtre est égale à $x[n]=x_1[n-\tau]$ ($\tau \in \mathbb{Z}$), alors la sortie sera égale à $y[n]=y_1[n-\tau]$.
\end{itemize}
\end{definition}
Un filtre LTI est entièrement décrit par sa réponse à une entrée de type impulsion ($x[n]=\delta[n]$). La réponse à une entrée de type impulsion est appelée réponse impulsionnelle et est notée $h[n]$. Lorsque l'entrée est un signal quelconque, la propriété suivante montre que la sortie du filtre s'obtient en convoluant l'entrée avec la réponse impulsionnelle du filtre.
\begin{propriete}[Sortie du filtre] \label{prop0} Lorsqu'un signal $x[n]$ est envoyé à l'entrée d'un filtre LTI, la sortie s'exprime sous la forme
\begin{align}
y[n]=x[n]*h[n] \triangleq \sum_{k=-\infty}^{\infty}x[k]h[n-k]
\end{align}
où $*$ désigne le produit de convolution et $h[n]$ correspond à la réponse impulsionnelle du filtre.
\end{propriete}
\begin{proof}
Voir annexe \ref{dprop0}.
\end{proof}

Une grande partie des filtres LTI peut s'exprimer à partir d'une équation aux différences
\begin{align}
y[n]=\sum_{m=0}^{M}b_m x[n-m]-\sum_{l=1}^{L}a_l y[n-l],\label{filtre1}
\end{align}
où les coefficients réels $b_m$ ($m=0,\cdots,M$) et $a_l$ ($l=1,\cdots,L$) correspondent respectivement aux coefficients de la partie non récursive et récursive du filtre. Notons que le coefficient multipliant $y[n]$ est implicitement égal à $a_0=1$. L'ordre du filtre est défini comme étant la plus grande valeur entre $M$ et $L$. 

La réponse impulsionnelle d'un filtre décrit par une équation aux différences s'obtient en posant $x[n]=\delta[n]$ dans l'équation (\ref{filtre1}). En fonction des coefficients $a_k$, il est alors possible de définir deux grandes catégories de filtres:
\begin{itemize}
\item les filtres à Réponse Impulsionnelle Finie (FIR) pour lesquels $a_l=0$ pour tout $l>0$,
\item les filtres à Réponse Impulsionnelle Infinie (IIR) pour lesquels il existe au moins une valeur de $l$ ($l>0$) telle que $a_l\ne 0$.
\end{itemize}

\subsection{Problématiques}

Dans ce document, nous allons traiter les problématiques suivantes.

\begin{itemize}
\item Connaissant la valeur des coefficients $b_m$ et $a_l$, est-il possible de comprendre l'effet d'un filtre sur une entrée quelconque $x[n]$  ? 
\item Est-il possible de déterminer les coefficients $b_m$ et $a_l$ d'un filtre à partir d'un ensemble de spécifications (fréquentielles ou temporelles) ?
\end{itemize}

\section{Analyse de Filtre}
\label{analyse}

Dans cette section, nous allons introduire les outils nécessaires pour l'analyse des filtres numériques. 

Pour illustrer le contenu de cette section, nous allons analyser le filtre d'ordre 2 suivant.
\begin{exemple}[Filtre 1] Le filtre 1 est décrit par l'équation aux différences 
\begin{align*}
y[n]&=0.065x[n]+0.13 x[n-1]+0.065x[n-2])\\
&+1.143y[n-1]-0.413y[n-2].
\end{align*}
\end{exemple}

\subsection{Analyse en $\mathcal{Z}$}

\begin{definition}[Transformée en $\mathcal{Z}$]
La transformée en $\mathcal{Z}$ d'une suite numérique $x[n]$ est définie par l'équation
\begin{align}
X(z)&\triangleq \sum_{n=-\infty}^{\infty}x[n]z^{-n}
\end{align}
où $z$ est une variable complexe.
\end{definition}
Il est important de noter que la transformée de $\mathcal{Z}$ d'un signal ne converge pas nécessairement quelque soit $z \in \mathbb{Z}$. Il est alors nécessaire de préciser la région de convergence (ROC) pour laquelle la série converge c-a-d les valeurs de $z$ telles que $|X(z)|<\infty$. A titre d'illustration, le tableau \ref{app_tabZ} présente les transformées en $\mathcal{Z}$ de plusieurs signaux et leurs regions de convergence associées.

\begin{table}[!t]
\begin{tabular}{ccc}
Signal  & Transformée en $\mathcal{Z}$ & ROC\\
\hline
$\delta[n]$ & $1$ & $z \in \mathbb{Z}$\\
$u[n]$& $\frac{1}{1-z^{-1}}$ & $|z| > 1$\\ 
$\delta[n-m]$& $z^{-m}$ & $z \ne 0$ \\ 
$a^n u[n]$& $\frac{1}{1-az^{-1}}$ & $|z| > |a|$ \\ 
$na^n u[n]$& $\frac{az^{-1}}{(1-az^{-1})^2}$ & $|z| > |a|$ \\ 
$\cos (\omega_0 n)u[n]$& $\frac{1-\cos(\omega_0)z^{-1}}{1-2\cos(\omega_0)z^{-1}+z^{-2}}$ & $|z| > 1$ \\ 
$\sin (\omega_0 n)u[n]$& $\frac{\sin(\omega_0)z^{-1}}{1-2\cos(\omega_0)z^{-1}+z^{-2}}$ & $|z| > 1$ \\ 
$r^n\cos (\omega_0 n)u[n]$& $\frac{1-r\cos(\omega_0)z^{-1}}{1-2r\cos(\omega_0)z^{-1}+r^2z^{-2}}$ & $|z| > r$ \\ 
$r^n\sin (\omega_0 n)u[n]$& $\frac{r\sin(\omega_0)z^{-1}}{1-2r\cos(\omega_0)z^{-1}+r^2z^{-2}}$ & $|z| > r$ \\ 
\hline
\end{tabular}
\caption{Quelques transformées en $\mathcal{Z}$}\label{app_tabZ}
\end{table}

\begin{propriete}[Transformée en $\mathcal{Z}$] La transformée en $\mathcal{Z}$ possède les propriétés suivantes (les réciproques étant également vraies):
\begin{enumerate}
\item Linéarité: Si $y[n]=\alpha x_1[n]+\beta x_2[n]$, alors $Y(z)=\alpha X_1(z)+\beta X_2(z)$.
\item Décalage temporel: Si $y[n]=x[n-\tau]$ ($\tau \in \mathbb{Z}$), alors $Y(z)=X(z)z^{-\tau}$.
\item Multiplication par une fonction exponentielle: Si $y[n]= a^n x[n]$, alors $ Y(z)=X(z/a)$.
\item Convolution: Si $y[n]= h[n]*x[n]$, alors $ Y(z)=H(z)X(z)$.
\item Théorème de la valeur finale: $\lim_{n\to \infty} x[n]=\lim_{z\to 1}(z-1)X(z)$.
\end{enumerate}
\end{propriete}

La propriété liée au décalage temporel indique qu'un retard d'un échantillon dans le domaine temporel revient à multiplier la transformée en $\mathcal{Z}$ par $z^{-1}$. En utilisant cette propriété, l'équation aux différences peut être représentée graphiquement par un schéma bloc où les blocs de fonction de transfert $z^{-1}$ introduisent un retard d'un échantillon. A titre d'exemple, la figure \ref{fig_schema} présente le schéma bloc du filtre~1.

\begin{figure}[!h]
\input{./tikz/block_diagram.tikz}
\caption{Schéma bloc du filtre 1.}\label{fig_schema}
\end{figure}

La propriété liée à la convolution montre l'importance de la transformée en $\mathcal{Z}$ de la réponse impulsionnelle, $H(z)$. Cette transformée en $\mathcal{Z}$ est appelé fonction de transfert du filtre. 
\begin{definition}[Fonction de Transfert]
La fonction de transfert d'un filtre correspond à la transformée en $\mathcal{Z}$ de sa réponse impulsionnelle c-a-d
\begin{align}
H(z)&=\sum_{n=-\infty}^{\infty}h[n]z^{-n}
\end{align}
\end{definition}

La propriété liée à la convolution montre que la fonction de transfert d'un filtre s'exprime également sous la forme $H(z)=Y(z)/X(z)$. Pour un filtre décrit par une équation aux différences, cette propriété permet d'exprimer la fonction de transfert du filtre en fonction des coefficients des parties récursive $a_l$ et non-recursive $b_m$ du filtre.
\begin{proposition}[Expression de la Fonction de transfert d'un filtre] La fonction de transfert d'un filtre numérique décrit par une équation aux différences s'exprime sous la forme ($a_0\triangleq 1$)
\begin{align*}
H(z)=\frac{\sum_{m=0}^{M}b_m z^{-m}}{\sum_{l=0}^{L}a_l z^{-l}}=\frac{B(z)}{A(z)}
\end{align*}\label{prop_ft}
\end{proposition}
\begin{proof}
Voir annexe \ref{dprop_ft}.
\end{proof}

A titre d'exemple, il est possible de montrer que la fonction de transfert du filtre 1 est donnée par 
\begin{align}
H(z)=\frac{0.065+0.13 z^{-1}+0.065z^{-2}}{1-1.143z^{-1}+0.413z^{-2}}.
\end{align}
%Dans le domaine en $\mathcal{Z}$, la sortie est égale à la fonction de transfert multipliée par l'entrée c-a-d $Y(z)=H(z)X(z)$. Par exemple, si l'entrée du filtre 1 est un échelon unitaire, la transformée en $\mathcal{Z}$ de la sortie sera égale à $Y(z)=H(z)/(1-z^{-1})$. En utilisant le théorème de la valeur finale, nous pouvons alors anticiper qu'en temporel la valeur finale sera égale à $\lim_{n\to\infty}y[n]=H(1)=1$. La figure \ref{fig_r_step} présente la réponse du filtre 1 lorsqu'un échelon unitaire est envoyé en entrée. Nous pouvons constater qu'effectivement la valeur finale est bien égale à~$1$.
%
%\begin{figure}[!t]
%\centering
%\begin{tikzpicture}\begin{axis}[height=0.8\linewidth,width=0.95\linewidth,xlabel=$n$,ylabel={$y[n]$},axis lines=center, enlargelimits=.1,ymin=-0.1,ymax=1.1,xmax=30]
%\addplot+[ycomb,color=black,mark options={fill=black}] table[x index=0,y index=2,col sep = comma] {./csv/r_step_0.txt};
%\end{axis}
%\end{tikzpicture}
%\caption{Réponse à un échelon unitaire (filtre 1).}\label{fig_r_step}
%\end{figure}

La figure \ref{fig_abs_z} présente le module de la fonction de transfert, $|H(z)|$, pour le filtre 1. Cette figure montre que le module de la fonction de transfert possède des "pics" et des "vallées". Pour mettre en évidence ces comportements singuliers, il est possible de réexprimer la fonction de transfert sous une forme factorisée. La forme factorisée donne explicitement les valeurs de $z$ pour lesquelles $H(z)$ tend vers $0$ (zéros) et les valeurs de $z$ pour lesquelles $H(z)$ tend vers l'infini (pôles). 

\begin{definition}[Pôles et Zéros] Une fonction de transfert peut s'exprimer sous forme factorisée de la manière suivante
\begin{align*}
H(z)=\left(\frac{b_0}{a_0}\right)\frac{\prod_{m=1}^{M}(1-c_m z^{-1})}{\prod_{l=1}^{L}(1-d_l z^{-1})}
\end{align*}
où les valeurs $c_m$ et $d_l$ correspondent respectivement aux zéros et aux pôles de la fonction de transfert.
\end{definition}

\begin{figure}[!t]
\centering
\includegraphics[width=1.1\linewidth]{./fig/z_plane_0.png}
\caption{Module de la Fonction de transfert, $|H(z)|$, pour le filtre 1.}
\label{fig_abs_z}
\end{figure}

En pratique, les pôles et les zéros s'obtiennent le plus souvent en utilisant des algorithmes numériques. A titre d'illustration, la fonction Python \texttt{root} permet d'établir que le filtre 1 possède un zéro double en $z=-1$ et deux pôles complexes conjugués en $z=0.57\pm 0.29j$. Notons que comme les coefficients $a_l$ et $b_m$ sont réels, les pôles et zéros complexes sont nécessairement conjugués. Il est courant de représenter la localisation des pôles et des zéros dans le plan complexe. Par convention, les pôles sont indiqués avec un $\times$ et les zéros avec un $\circ$. La figure \ref{fig_carte_pz} présente la localisation des pôles et des zéros pour le filtre 1.

\begin{figure*}[t]
\begin{subfigure}[b]{0.5\textwidth}
\centering
\begin{tikzpicture}\begin{axis}[height=0.8\linewidth,width=0.95\linewidth,unit vector ratio = 1 1,xtick={-1,0, 1},ytick={-1,0, 1},ymin=-1.5,ymax=1.5,xmin=-1.5,xmax=1.5,axis lines=center, enlargelimits=.1,xlabel={$\Re e(z)$}, ylabel={$\Im m(z)$}]
\addplot [color=black,mark=x,mark options={scale=2, fill=black}] coordinates {(0.57149025,-0.2935992)};
\addplot [color=black,mark=x,mark options={scale=2, fill=black}] coordinates {(0.57149025,0.2935992)};
\addplot [color=black,mark=o,mark options={scale=2}] coordinates {(-1,0)};
\addplot [dashed,domain=0:2*pi,samples=50]({cos(deg(x))},{sin(deg(x))});
\end{axis}\end{tikzpicture}
\caption{Filtre 1a: Cartographie des Pôles et des Zéros}
\label{fig_carte_pz}
\end{subfigure}%
\begin{subfigure}[b]{0.5\textwidth}
\centering
\begin{tikzpicture}\begin{axis}[height=0.8\linewidth,width=0.9\linewidth,axis lines=center, enlargelimits=.1,ylabel={$h[n]$},xlabel={$n$}]
\addplot+[ycomb,color=black,mark options={fill=black}] table[x expr=\coordindex,y index=0] {./csv/ri_0.txt};
\end{axis}
\end{tikzpicture}
\caption{Filtre 1a: Réponse Impulsionnelle}
\end{subfigure}
\vskip 1em
\begin{subfigure}[b]{0.5\textwidth}
\centering
\begin{tikzpicture}\begin{axis}[height=0.8\linewidth,width=0.95\linewidth,unit vector ratio = 1 1,xtick={-1,0, 1},ytick={-1,0, 1},ymin=-1.5,ymax=1.5,xmin=-1.5,xmax=1.5,axis lines=center, enlargelimits=.1,xlabel={$\Re e(z)$}, ylabel={$\Im m(z)$}]
\addplot [color=black,mark=x,mark options={scale=2, fill=black}] coordinates {(-1.4,0)};
\addplot [color=black,mark=x,mark options={scale=2, fill=black}] coordinates {(0.57149025,-0.2935992)};
\addplot [color=black,mark=x,mark options={scale=2, fill=black}] coordinates {(0.57149025,0.2935992)};
\addplot [color=black,mark=o,mark options={scale=2}] coordinates {(0,0)};
\addplot [color=black,mark=o,mark options={scale=2}] coordinates {(-1,0)};
\addplot [dashed,domain=0:2*pi,samples=50]({cos(deg(x))},{sin(deg(x))});
\end{axis}\end{tikzpicture}
\caption{Filtre 1b: Cartographie des Pôles et des Zéros}
\end{subfigure}%
\begin{subfigure}[b]{0.5\textwidth}
\centering
\begin{tikzpicture}\begin{axis}[height=0.75\linewidth,width=0.9\linewidth,axis lines=center, enlargelimits=.1,ylabel={$h[n]$},xlabel={$n$}]
\addplot+[ycomb,color=black,mark options={fill=black}] table[x expr=\coordindex,y index=0] {./csv/ri_1.txt};
\end{axis}
\end{tikzpicture}
\caption{Filtre 1b: Réponse Impulsionnelle}
\end{subfigure}
\caption{Lien entre la localisation des pôles et zéros et la réponse impulsionnelle.\label{fig_pz_imp}}
\end{figure*}

La localisation des pôles joue un rôle de premier plan sur la propriété de stabilité du filtre. De manière formelle, un filtre est dit stable si sa réponse impulsionnelle est absolument sommable c-a-d $\sum_{n=-\infty}^{\infty}|h[n]|<\infty$. La propriété suivante montre que cette propriété est directement liée au module des pôles de la fonction de transfert.
\begin{propriete}[Stabilité] Un filtre est stable si tous les pôles de sa fonction de transfert sont inclus dans le cercle de rayon unité c-a-d si pour tout $l$ 
\begin{align}
|d_l|\le 1.
\end{align}
\label{prop1}
\end{propriete}
\begin{proof}
Voir annexe \ref{dprop1}.
\end{proof}
La figure \ref{fig_pz_imp} présente la localisation des pôles et des zéros ainsi que la réponse impulsionnelle de deux filtres IIR. Le premier filtre est stable car tous ses pôles sont inclus dans le cercle de rayon unité. A l'inverse, le second filtre est instable car il possède un pôle en $z=-1.4$ dont le module est supérieur à 1. Pour ce second filtre, nous constatons que la réponse impulsionnelle semble tendre vers des valeurs infinies.

\subsection{Analyse Fréquentielle}
Le domaine fréquentielle est très utilisé pour analyser le comportement des filtres LTI. Le succès de ce domaine d'analyse est principalement lié au fait que la réponse d'un filtre à une exponentielle complexe de pulsation $\omega$ est une exponentielle complexe de même pulsation, multipliée par un coefficient complexe. Ce coefficient multiplicateur s'obtient à partir de la transformée de Fourier de la réponse impulsionnelle.

\begin{definition}[Transformée de Fourier]
\label{prop_f}
Une suite numérique $x[n]$ peut se décomposer sous la forme 
\begin{align}
x[n]&= \frac{1}{2\pi}\int_{-\pi}^{\pi} X(e^{j\omega})e^{j\omega n}d\omega \label{e_def_if}
\end{align}
où la quantité $X(e^{j\omega})$ correspond à la transformée de Fourier (à temps discret) de $x[n]$. Cette quantité est définie par
\begin{align}
X(e^{j\omega})&\triangleq \sum_{n=-\infty}^{\infty}x[n]e^{-j\omega n}.
\end{align}
\end{definition}
\begin{proof}
Voir annexe \ref{dprop_f}.
\end{proof}

\begin{proposition}[Lien Transformée en $\mathcal{Z}$ et Transformée de Fourier]
La transformée de Fourier s'obtient en évaluant la transformée en $\mathcal{Z}$ sur le cercle de rayon unité c-a-d $z=e^{j\omega}$.
\end{proposition}


Comme $X(e^{j\omega})=X(e^{j(\omega+2k\pi)})$ ($k \in \mathbb{Z}$), la transformée de Fourier à temps discret est $2\pi$-périodique. De plus lorsque $x[n]\in \mathbb{R}$, il est possible de démontrer que $X(e^{j\omega})=X^{*}(e^{-j\omega})$ (symétrie hermitienne). Pour ces deux raisons, la transformée de Fourier est couramment représentée dans l'intervalle $[0,\pi]$.



\begin{figure*}[t]
\begin{subfigure}[b]{0.5\textwidth}
\centering
\begin{tikzpicture}
\begin{axis}[height=0.8\linewidth,width=0.9\linewidth,axis lines=center, enlargelimits=.1,ylabel={$|H(e^{j\omega})|$},xlabel={$\omega$}]
\addplot+[mark=none,color=black] table[x index=0,y index=1,col sep = comma] {./csv/rf_0.txt};
\end{axis}
\end{tikzpicture}
\caption{Module}
\label{fig_f_abs}
\end{subfigure}%
\begin{subfigure}[b]{0.5\textwidth}
\centering
\begin{tikzpicture}
\begin{axis}[height=0.8\linewidth,width=0.9\linewidth,axis lines=center, enlargelimits=.1,ylabel={$Arg[H(e^{j\omega})]$},xlabel={$\omega$}]
\addplot+[mark=none,color=black] table[x index=0,y index=2,col sep = comma] {./csv/rf_0.txt};
\end{axis}
\end{tikzpicture}
\caption{Argument }
\end{subfigure}
\caption{Réponse fréquentielle (filtre 1).}
\label{fig_z2f}
\end{figure*}

Pour un filtre LTI, la transformée de Fourier de la sortie s'obtient en multipliant la transformée de Fourier de la réponse impulsionnelle par la transformée de Fourier de l'entrée c-a-d $Y(e^{j\omega})=H(e^{j\omega})X(e^{j\omega})$. La transformée de Fourier de la réponse impulsionnelle est appelée réponse fréquentielle du filtre.

\begin{definition}[Réponse fréquentielle]
La réponse fréquentielle d'un filtre correspond à la transformée de Fourier de sa réponse impulsionnelle c-a-d
\begin{align}
H(e^{j\omega})=\sum_{n=-\infty}^{\infty}h[n]e^{-j\omega n}
\end{align}
\end{definition}
La réponse frequentielle $H(e^{j\omega})$ est généralement une quantité complexe qu'il est possible d'analyser via son module et son argument.
\begin{proposition}[Module / Argument de la réponse fréquentielle]
La fonction de transfert $H(e^{j\omega})$ peut se décomposer sous la forme
\begin{align}
H(e^{j\omega})=|H(e^{j\omega})| e^{j Arg[H(e^{j\omega})]},
\end{align} 
où $|H(e^{j\omega})|$ et $Arg[H(e^{j\omega})]$ désignent respectivement le module et l'argument. 
\end{proposition}

Pour analyser le comportement d'un filtre, il est utile de représenter le module et l'argument de la réponse fréquentielle en fonction de $\omega$. L'affichage du module et de l'argument permet d'avoir une interprétation concrète de l'effet du filtre sur une entrée quelconque. En effet à la pulsation $\omega$, le filtre va appliquer un gain $|H(e^{j\omega})|$ et un déphasage (retard) $Arg[H(e^{j\omega})]$. A titre d'illustration, la figure \ref{fig_z2f} présente le module et l'argument de la réponse fréquentielle du filtre 1. Nous observons que le filtre se comporte comme un filtre passe-bas. Notons que lorsque l'argument est égal à $Arg[H(e^{j\omega})]=-\omega \tau$, ce qui n'est pas le cas ici, le filtre est dit à phase linéaire. En pratique, cette propriété est souvent recherchée car elle évite la présence de distorsion de phase.
%A titre d'illustration, la figure \ref{fig_r_sine} présente la réponse temporel du filtre lorsque qu'une sinusoïde de pulsation $\omega_0=0.7$ rad/ech est appliqué en entrée. A cette pulsation, la réponse fréquentielle du filtre indique que le filtre applique un gain $\approx$ 0.6 et un déphasage de $-1.7$ radian ($\approx 2.3$ échantillons). Ces valeurs se vérifient également à partir de la réponse temporel du filtre.
%
%\begin{figure*}
%\begin{subfigure}[b]{0.5\linewidth}
%\centering
%\begin{tikzpicture}\begin{axis}[height=0.8\linewidth,width=0.95\linewidth,xlabel=$n$,ylabel={$x[n]$},axis lines=center, enlargelimits=.1,ymin=-1.1,ymax=1.1,xmax=30]
%\addplot+[ycomb,color=black,mark options={fill=black}] table[x index=0,y index=1,col sep = comma] {./csv/r_sine_0.txt};
%\end{axis}
%\end{tikzpicture}
%\caption{Entrée: $x[n]$.}
%\end{subfigure}%
%\begin{subfigure}[b]{0.5\linewidth}
%\centering
%\begin{tikzpicture}\begin{axis}[height=0.8\linewidth,width=0.95\linewidth,xlabel=$n$,ylabel={$y[n]$},axis lines=center, enlargelimits=.1,ymin=-1.1,ymax=1.1,xmax=30]
%\addplot+[ycomb,color=black,mark options={fill=black}] table[x index=0,y index=2,col sep = comma] {./csv/r_sine_0.txt};
%\end{axis}
%\end{tikzpicture}
%\caption{Sortie: $y[n]$.}
%\end{subfigure}
%\caption{Réponse à une sinusoïde de pulsation $\omega_0=0.7$ rad/éch (filtre 1).}\label{fig_r_sine}
%\end{figure*}


Pour les filtres décrits par une équation aux différences, la réponse fréquentielle peut s'exprimer en fonction de la transformée de Fourier des coefficients $a_l$ et $b_m$ du filtre.
\begin{proposition}[Réponse Frequentielle]
La réponse fréquentielle d'un filtre s'exprime sous la forme
\begin{align*}
H(e^{j\omega})\triangleq \frac{Y(e^{j\omega})}{X(e^{j\omega})}=\frac{\sum_{k=0}^{M}b_m e^{-j\omega k}}{\sum_{l=0}^{N}a_l e^{-j\omega l}}=\frac{B(e^{j\omega})}{A(e^{j\omega})}
\end{align*}
\end{proposition}
Le module et l'argument s'expriment alors sous la forme suivante.
\begin{proposition}[Module et Argument de la Réponse Frequentielle]
Le module est l'argument de la réponse fréquentielle s'expriment sous la forme
\begin{align*}
|H(e^{j\omega})|&=\frac{|B(e^{j\omega})|}{|A(e^{j\omega})|}\\
Arg[H(e^{j\omega})]&=Arg[B(e^{j\omega})]-Arg[A(e^{j\omega})]
\end{align*}
\end{proposition}

\subsection{Comparaison des filtres FIR et IIR}

En utilisant les résultats précédents, il est possible de démontrer que les filtres FIR et IIR possèdent des propriétés bien différentes.
\begin{itemize}
\item \textbf{Filtres FIR}. Comme $a_l=0$ pour tout $l\ne 0$, les filtres FIR n'ont pas de partie récursive. Ces filtres n'ont pas de pôle et sont donc nécessairement stables. Dans le cas où les coefficients du filtre sont symétriques, c-a-d $b_m=b_{M-k}$, il est possible de démontrer que la phase est linéaire, c-a-d $Arg[H(e^{j\omega})]=-\omega \tau$. L'inconvénient majeur des filtres FIR réside dans la pente de $|H(e^{j\omega})|$. En effet, il est nécessaire d'utiliser un ordre $M$ élevé pour obtenir un filtre avec une pente élevée.
\item \textbf{Filtres IIR}. Les filtres IIR possèdent une partie récursive et donc un ou des pôles $d_l$. Un filtre IIR est stable si tous ses pôles sont inclus dans le cercle de rayon unité. La phase d'un filtre IIR ne peut pas être linéaire et est généralement difficile à contrôler. Par rapport à des filtres FIR de même ordre, les filtres IIR présentent l'avantage d'avoir des pentes nettement plus élevées. 
\end{itemize}
L'ensemble de ces propriétés est résumé dans le tableau~\ref{t_comp_filtres}.
\begin{table}
\centering
\begin{tabular}{ccc}
& FIR & IIR \\
\hline
Partie récursive  & Non & Oui\\
Pôles & Non & Oui \\
Stable & Oui & Oui si $|d_l|<1$\\
Phase Linéaire & Oui si  $b_m=b_{M-k}$& Non\\
Ordre & Elevé & Faible \\
\hline
\end{tabular}
\caption{Comparaison des filtres FIR et IIR}\label{t_comp_filtres}
\end{table}



\section{Synthèse par numérisation de filtres analogiques}

Pour concevoir un filtre numérique IIR, une technique possible consiste à convertir un filtre analogique. En utilisant cette approche, nous allons pouvoir réutiliser un large panel de techniques initialement développées dans le contexte analogique. Dans le domaine analogique, un filtre est décrit par sa fonction de transfert, $H_a(s)$, où $s$ désigne la variable de Laplace. Dans les sous-sections suivantes, nous allons présenter deux techniques permettant d'approcher la réponse fréquentielle d'un filtre analogique par un filtre numérique. 

Pour illustrer notre section, nous allons utiliser comme référence le filtre analogique suivant.

\begin{exemple}[Filtre 2] Le filtre analogique 2 est défini par la fonction de transfert suivante
\begin{align*}
H_a(s)&=\frac{10000}{(s-100e^{3j\pi/4})(s-100e^{-3j\pi/4})}
\end{align*}
\end{exemple}
La pulsation de coupure du filtre est égale à $\omega_a=100$ rad/s ($20\log_{10}(|H(\omega_a)|)=-3$dB). Ce filtre sera échantillonné à une fréquence d'échantillonnage $F_e=1$kHz ($T_e=0.001$s).

\subsection{Invariance Impulsionnelle}

\begin{definition}[Invariance Impulsionnelle] La technique de synthèse basée sur l'invariance impulsionnelle est décrite par la procédure suivante
\begin{enumerate}
\item Calcul de la réponse impulsionnelle à temps continu: $h_a(t)=\mathcal{L}^{-1}\{H_a(s)\}$ où $\mathcal{L}^{-1}\{.\}$ désigne la transformée de Laplace inverse\footnote{Le plus souvent une décomposition en éléments simples sera nécessaire avant la transformation inverse pour obtenir une somme de formes canoniques}.
\item Discrétisation de la réponse impulsionnelle: $h[n]=T_e h_a(n T_e)$,  où $T_e$ désigne la période d'échantillonnage.
\item Détermination de la fonction de transfert en $\mathcal{Z}$, $H(z)=\mathcal{Z}\{h[n]\}$ via le tableau \ref{app_tabZ}.
\item Obtention de l'équation aux différences à partir de $H(z)$.
\end{enumerate}
\end{definition}

Dans le cas général, la procédure de synthèse peut être fastidieuse. Toutefois, lorsque la fonction de transfert $H_a(s)$ contient uniquement des pôles de premier ordre, la proposition suivante montre qu'il est possible d'obtenir rapidement l'expression de $H(z)$.
\begin{proposition}[Pôles de premier ordre]
\label{prop_ii}
Lorsque $H_a(s)=\sum_{k=1}^{N} \frac{A_k}{s-s_k}$, la fonction de transfert en $\mathcal{Z}$ du filtre s'exprime sous la forme
\begin{align}
H(z)&=\sum_{k=1}^{N} \frac{T_e A_k}{1-e^{s_kT_e}z^{-1}}
\end{align}
\end{proposition}
\begin{proof}
Voir Annexe \ref{a_prop_ii}.
\end{proof}

La proposition \ref{prop_ii} montre que la synthèse d'un filtre ne contenant que des pôles de premier ordre s'obtient en réalisant un mapping des pôles $s_k \to e^{s_kT_e}$ et en appliquant le coefficient multiplicateur $T_e$. A titre d'illustration, l'utilisation d'une decomposition en éléments simples permet d'écrire le filtre 2 sous la forme

\begin{align*}
H_a(s)&=\frac{A_1}{s-s_1}+\frac{A_2}{s-s_2}
\end{align*}
où $A_1=-A_2=-100j/\sqrt{2}$ et $s_1=s_2^*=100e^{3j\pi/4}$. En utilisant la proposition \ref{prop_ii}, ce filtre peut être approché en numérique par le filtre de fonction de transfert
\begin{align*}
H(z)&=\frac{0.00931z^{-1}}{1-1.8588z^{-1}+0.8681z^{-2}}
\end{align*}
La réponse fréquentielle de ce filtre est présentée sur la figure \ref{fig_comp_iir}. Notons que l'inconvenient majeur de cette technique de synthèse réside dans l'apparition d'un phénomène de repliement du spectre.

\subsection{Transformée Bilinéaire}
Dans le domaine analogique, un filtre est décrit par une équation différentielle. Pour numériser un filtre analogique, une approche possible consiste à utiliser une technique d'intégration numérique basée sur la méthode des trapèzes. Cette approche est nommée transformée bilinéaire. En pratique, la transformée bilinéaire s'obtient simplement en réalisant la substitution $s=c\frac{1-z^{-1}}{1+z^{-1}}$, où $c$ correspond à un paramètre libre du filtre. 

\begin{proposition}[Transformée Bilinéaire] En utilisant la transformée Bilinéaire, la fonction de transfert en $\mathcal{Z}$ d'un filtre numérique s'obtient à partir de la fonction de transfert analogique $H_a(z)$ sous la forme
\begin{align}
H(z)&=H_a\left(c\frac{1-z^{-1}}{1+z^{-1}}\right)
\end{align}
\end{proposition}
Un point appartenant à l'axe $s=j\omega$ dans le plan de Laplace est déplacé en un point appartenant au cercle de rayon unité $e^{j\omega}$ dans le domaine en $\mathcal{Z}$. De plus, il est démontré dans l'annexe \ref{ap_tf}, qu'après transformation, la pulsation $\omega_a$ dans le domaine continue est déplacée à la pulsation $\omega=\frac{2}{T_e}\textrm{atan}(\omega_a/c)$ dans le domaine numérique. Ce \textit{warping frequentiel} non-linéaire est illustré par la figure \ref{fig_warp} pour $c=T_e=1$. 

\begin{figure}
\centering
\begin{tikzpicture}
\begin{axis}[xmin=-10,xmax=10,axis lines=center, enlargelimits=.1,xlabel={$\omega_a$}, ylabel={$\omega$}]
\addplot [domain=-10:10,samples=50]{2*rad(atan(x))};
\addplot [dashed] coordinates { (-10,-3.14) (10,-3.14) } node [above]{$-\pi$};
\addplot [dashed] coordinates { (-10,3.14) (10,3.14) } node [below]{$\pi$};
\end{axis}
\end{tikzpicture}
\caption{Transformée Bilinéaire: warping fréquentiel}\label{fig_warp}
\end{figure}

Le paramètre $c$ est le plus souvent fixé pour faire en sorte que la réponse fréquentielle du filtre numérique corresponde à la réponse fréquentielle du filtre analogique à une pulsation particulière $\omega_0$.

\begin{proposition}
A la pulsation $\omega_0$, les réponses fréquentielles des filtres analogique et numérique sont équivalentes si et seulement si $c=\omega_0 / \tan(\omega_0 T_e/2)$.
\end{proposition}

A titre d'illustration, pour obtenir équivalence à la pulsation $\omega_0$ entre le filtre 2 et sa version numérisée, la paramètre libre doit être fixé à $c=1998.33$. En appliquant la transformée bilinéaire, nous obtenons alors le filtre numérique décrit par la fonction de transfert en $\mathcal{Z}$ suivante:

\begin{align}
H(z)&=\frac{0.00233+0.00466z^{-1}+ 0.00233z^{-2}}{1-1.85879z^{-1}+ 0.86812z^{-2}}
\end{align}
La réponse fréquentielle de ce filtre est présentée sur la figure \ref{fig_comp_iir}.

\begin{figure}[t]
\centering
\begin{tikzpicture}
\begin{semilogxaxis}[xmin=10,ymin=-80,xmax=3141,xmajorgrids,ymajorgrids,ylabel={$|H(e^{j\omega})|_{dB}$},xlabel={$\omega$ (rad/s)},legend pos=south west]
\addplot+[mark=none,color=black] table[x index=0,y index=1,col sep = comma] {./csv/iir_comp.txt};
\addplot+[mark=none,blue,color=red] table[x index=0,y index=2,col sep = comma] {./csv/iir_comp.txt};
\addplot+[mark=none,color=blue] table[x index=0,y index=3,col sep = comma] {./csv/iir_comp.txt};
\addplot [dashed] coordinates { (10,-3) (10000,-3) } node[above]{$-3$dB} ;
\legend{$H_a(s)$,$H(z)$ (invariance),$H(z)$ (bilinéaire)}
\end{semilogxaxis}
\end{tikzpicture}
\caption{Comparaison de deux techniques de synthèse: invariance impulsionnelle et transformée bilinéaire.}\label{fig_comp_iir}
\end{figure}

\section{Synthèse par modèle de réponse impulsionnelle ou fréquentielle}
\label{s_design_ri}
Un filtre numérique est décrit par en ensemble de coefficients $b_m$ et $a_k$. Pour synthétiser un filtre, une technique possible consiste à minimiser un critère d'erreur entre la réponse attendue (impulsionnelle ou fréquentielle) et la réponse du filtre. Un critère d'erreur couramment utilisé est l'erreur quadratique. Dans cette partie, nous allons voir comment fixer les coefficients $b_m$ et $a_l$ afin de minimiser l'erreur quadratique entre la réponse impulsionnelle attendue et la réponse impulsionnelle du filtre c-a-d
\begin{align}
\epsilon&=\sum_{n=-\infty}^{\infty}(h_d[n]-h[n])^2\label{eq_ref_h}
\end{align}
Notons que la réponse impulsionnelle désirée $h_d[n]$ est parfois donnée implicitement via la réponse fréquentielle, $H_d(e^{j\omega})$. Dans le cas, il est nécessaire de calculer $h_d[n]$ via l'expresssion (\ref{e_def_if}).

Pour illustrer cette section, nous allons synthétiser le filtre suivant.
\begin{exemple}[Filtre 3]
Le filtre 3 est un filtre passe-bas en mur de brique (pente infinie) comportant une phase linéaire. La réponse fréquentielle de ce filtre est donnée par $H_d(e^{j\omega})=|H_d(e^{j\omega})|e^{-j\omega N/2}$ où
\begin{align}
|H_d(e^{j\omega})|&=\left\{\begin{array}{cl}
1&\textrm{, pour } |\omega| \le \omega_a T_e\\
0 &\textrm{, ailleurs}\end{array}\right.
\end{align}
avec $\omega_a=100$ rad/s et $T_e=0.001$s.
\end{exemple}
La réponse impulsionnelle du filtre 3 s'obtient en utilisant l'équation (\ref{e_def_if}). Après quelques calculs, nous obtenons
\begin{align}
h_d[n]&=\frac{\sin\left(\omega_a T_e (n-N/2)\right)}{\pi (n-N/2)}
\end{align}
La réponse impulsionnelle de ce filtre est présentée dans la figure \ref{fig_fir_ri} pour $N=200$ échantillons.
\begin{figure}[t]
\centering
\begin{tikzpicture}
\begin{axis}[xmin=0,xmax=200,axis lines=center, enlargelimits=.1,ylabel={$h_d[n]$},xlabel={$n$},legend pos=south west]
\addplot+[mark=none,color=black] table[x index=0,y index=1,col sep = comma] {./csv/fir_w_ri.txt};
\end{axis}
\end{tikzpicture}
\caption{Réponse Impulsionnelle du filtre 3.}\label{fig_fir_ri}
\end{figure}

\subsection{Filtres FIR (Technique par fenêtrage)}

Mathématiquement, un filtre FIR d'ordre $M=N$ est modélisé par la relation
\begin{align}
y[n]=\sum_{m=0}^{M} b_m x[n-m]
\end{align}
La réponse impulsionnelle d'un filtre FIR est donnée par
\begin{align}
h[n]=\left\{\begin{array}{cl}b_n&\textrm{ pour }0\le n \le M\\
0& \textrm{ ailleurs}
\end{array}\right.
\end{align}
Dans ce contexte, la minimisation de l'erreur quadratique $\epsilon$ dans (\ref{eq_ref_h}) s'obtient facilement en imposant l'égalité $b_m=h_d[m]$ dans l'intervalle $0\le m \le M$. Cette technique est couramment appelée technique de synthèse par fenêtrage car la réponse impulsionnelle obtenue correspond à la réponse impulsionnelle attendue multipliée par une fenêtre rectangulaire. Notons qu'il est possible de généraliser cette technique en remplaçant la fenêtre rectangulaire par une fenêtre de pondération $w[m]$ afin de limiter l'effet de la troncature en temps.
\begin{proposition}[Technique de synthèse par fenêtrage]
La technique de synthèse par fenêtrage s'obtient en fixant 
\begin{align}
\widehat{b}_m=\left\{\begin{array}{cl}w[m]h_d[m]&\textrm{ pour }0\le m \le M\\
0& \textrm{ ailleurs}
\end{array}\right.
\end{align}
où $w[m]$ est une fenêtre de pondération.
\end{proposition}

La figure \ref{fig_comp_fir} présente l'allure de la réponse fréquentielle obtenue pour la synthèse du Filtre 3. Cette figure compare les résultats obtenus avec trois fenêtres de pondération différentes (rectangulaire, Hamming et Blackman). Nous pouvons observer un compromis entre la raideur de la pente du filtre et l'amplitude des lobes secondaires.
\begin{figure}[t]
\centering
\begin{tikzpicture}
\begin{semilogxaxis}[xmin=10,ymin=-80,xmax=3141,xmajorgrids,ymajorgrids,ylabel={$|H(e^{j\omega})|_{dB}$},xlabel={$\omega$ (rad/s)},legend pos=south west]
\addplot+[mark=none,color=blue] table[x index=0,y index=1,col sep = comma] {./csv/fir_w_rf.txt};
\addplot+[mark=none,blue,color=red] table[x index=0,y index=2,col sep = comma] {./csv/fir_w_rf.txt};
\addplot+[mark=none,color=black] table[x index=0,y index=3,col sep = comma] {./csv/fir_w_rf.txt};
\legend{Rectangulaire,Hamming, Blackman}
\end{semilogxaxis}
\end{tikzpicture}
\caption{Technique de synthèse par fenêtrage. Influence de la fenêtre de pondération ($M=200$)}\label{fig_comp_fir}
\end{figure}


\subsection{Filtre IIR}
Mathématiquement, un filtre IIR est modélisé par la relation
\begin{align}
y[n]=\sum_{k=0}^{M} b_m x[n-m]-\sum_{l=1}^{L}a_l y[n-l]
\end{align}
La réponse impulsionnelle d'un filtre IIR s'obtient par récurrence via l'expression
\begin{align}
h[n]=\sum_{k=0}^{M} b_m \delta[n-m]-\sum_{l=1}^{L}a_l h[n-l]
\end{align}
La technique de synthèse optimale consiste à minimiser l'erreur quadratique $\epsilon$ par rapport aux coefficients de la partie récursive et non-récursive du filtre. Cette technique de synthèse est toutefois difficile à mettre en oeuvre car elle nécessite de minimiser une fonction multimodale dans un espace de dimension $L+M+1$. 

Pour simplifier le problème, une technique alternative consiste à minimiser, au lieu de l'erreur $\epsilon$, l'erreur quadratique \textit{tronquée et filtrée} définie par
\begin{align}
\epsilon_f&=\sum_{n=0}^{N-1}(h_{df}[n]-b[n])^2\label{eq_ref_h2}
\end{align}
où $h_{df}[n]=\sum_{l=0}^{L}a_l h_d[n-l]$ et $a_0=1$. Lorsque $N$ est suffisamment grand, cette technique permet d'obtenir une réponse impulsionnelle proche de la réponse impulsionnelle attendue. Cette approche est couramment appelée méthode de Prony en référence à son inventeur. En utilisant cette approche, il est démontré en annexe \ref{ap_prony} que les coefficients $a_l$ et $b_m$ s'obtiennent de la manière suivante.

\begin{proposition}[Technique de synthèse par la méthode de Prony]\label{prop_prony}
La technique de synthèse par la méthode Prony s'obtient en imposant les coefficients $\widehat{\textbf{a}}=[\widehat{a}_1,\cdots,\widehat{a}_L]^T$ et $\widehat{\textbf{b}}=[\widehat{b}_0,\cdots,\widehat{b}_M]^T$ suivants
\begin{align}
\widehat{\textbf{a}}&=(\textbf{H}^T\textbf{H})^{-1}\textbf{H}^T\textbf{x}\\
\widehat{\textbf{b}}&=\textbf{h}+\textbf{H}_0\widehat{\textbf{a}}
\end{align}
où
\begin{align*}
\textbf{H}_0&=\begin{bmatrix} 0&0&\cdots&0\\
h_d[0]&0&\cdots&0\\
h_d[1]&h_d[0]&\cdots&0\\
\vdots& &\cdots&\vdots\\
h_d[M-1]&h_d[M-2]&\cdots&h_d[M-L]\\
\end{bmatrix}\\
\textbf{H}&=-\begin{bmatrix} h_d[M]&h_d[M-1]&\cdots&h_d[M-L+1]\\
h_d[M+1]&h_d[M]&\cdots&h_d[M-L+2]\\
\vdots& &\cdots&\vdots\\
h_d[N-2]&h_d[N-3]&\cdots&h_d[N-1-L]\\
\end{bmatrix}\\
\textbf{x}&=\begin{bmatrix} h_d[M+1]&h_d[M+2]&\cdots &h_d[N-1]\end{bmatrix}^T\\
\textbf{h}&=\begin{bmatrix} h_d[0]&h_d[1]&h_d[2]&\cdots &h_d[M]\end{bmatrix}^T
\end{align*}
\end{proposition}

La figure \ref{fig_iir_prony} présente la réponse fréquentielle obtenue avec la méthode de Prony pour la synthèse du Filtre 3 avec un filtre IIR d'ordre $6$. Même si cette technique de sythèse est relativement simple, nous pouvons constater des oscillations importantes en basse-fréquence. 

\begin{figure}[t]
\centering
\begin{tikzpicture}
\begin{semilogxaxis}[xmin=10,ymin=-80,xmax=3141,xmajorgrids,ymajorgrids,ylabel={$|H(e^{j\omega})|_{dB}$},xlabel={$\omega$ (rad/s)},legend pos=south west]
\addplot+[mark=none,color=blue] table[x index=0,y index=1,col sep = comma] {./csv/iir_prony_rf.txt};
\end{semilogxaxis}
\end{tikzpicture}
\caption{Technique de synthèse de filtres par la méthode de Prony ($M=L=6$, $N=200$)}\label{fig_iir_prony}
\end{figure}



 
\appendices

\section{Quelques signaux numériques}
\label{app_sig}
La figure \ref{fig_sig} présente l'allure de quelques signaux numériques. Ces signaux sont définis mathématiquement sous la forme:
\begin{itemize}
\item Impulsion:
\begin{align*}
\delta[n]&=\left\{\begin{array}{cc} 1&\textrm{si }n=0,\\
0 &\textrm{ailleurs }\end{array}\right.
\end{align*}
\item Echelon:
\begin{align*}
u[n]&=\left\{\begin{array}{cc} 1&\textrm{si }n\ge 0,\\
0 &\textrm{ailleurs }\end{array}\right.
\end{align*}
\item Exponentielle:
\begin{align*}
x[n]&=a^{n} u[n]
\end{align*}
\item Sinusoide:
\begin{align*}
x[n]&=a \sin(\omega n+\phi) u[n]
\end{align*}
\end{itemize}


\begin{figure}[!h]
\centering
\begin{subfigure}[b]{0.45\textwidth}
\centering
\begin{tikzpicture}
\begin{axis}[height=0.6\textwidth,width=0.95\textwidth,domain = -2:10,samples = 13,ymin=-0,ymax=1.2,axis lines=center, enlargelimits=.1,ticks=none,xlabel=$n$, ylabel={$\delta[n]$}]
\addplot+[ycomb,black,mark options={fill=black}] {(x>-1)*(x<1)};
\end{axis}
\end{tikzpicture}
\caption{Impulsion: $\delta[n]$}
\end{subfigure}

\begin{subfigure}[b]{0.45\textwidth}
\centering
\begin{tikzpicture}
\begin{axis}[height=0.6\textwidth,width=0.95\textwidth,domain = -2:10,samples = 13,ymin=-0,ymax=1.2,axis lines=center, enlargelimits=.1,ticks=none,xlabel=$n$, ylabel={$u[n]$}]
\addplot+[ycomb,black,mark options={fill=black}] {x>-0.5};
\end{axis}
\end{tikzpicture}
\caption{Echelon: $u[n]$}
\end{subfigure}

\begin{subfigure}[b]{0.45\textwidth}
\centering
\begin{tikzpicture}
\begin{axis}[height=0.6\textwidth,width=0.95\textwidth,domain = -2:10,samples = 13,ymin=-0,ymax=1.2,axis lines=center, enlargelimits=.1,ticks=none,xlabel=$n$, ylabel={$x[n]$}]
\addplot+[ycomb,black,mark options={fill=black}] {(0.5^x)*(x>-1)};
\end{axis}
\end{tikzpicture}
\caption{Exponentielle : $x[n]$}
\end{subfigure}

\begin{subfigure}[b]{0.45\textwidth}
 \centering
\begin{tikzpicture}
\begin{axis}[height=0.6\textwidth,width=0.95\textwidth,domain = -2:10,samples = 13,ymin=-1.2,ymax=1.2,axis lines=center, enlargelimits=.1,ticks=none,xlabel=$n$, ylabel={$x[n]$}]
\addplot+[ycomb,black,mark options={fill=black}] {(cos(30*x)*(x>-1)};
\end{axis}
\end{tikzpicture}
\caption{Sinusoide : $x[n]$}
\end{subfigure}
\caption{Allure des signaux usuels}
\label{fig_sig}
\end{figure}



\section{Démonstration de la proposition \ref{prop0}}
\label{dprop0}

Notons $y[n]=T_n\{x[.]\}$ le $n^{ieme}$ échantillon en sortie d'un filtre lorsqu'un signal $x[n]$ ($n\in \mathbb{Z}$) est appliqué en entrée. L'entrée du filtre peut être décomposé sous la forme $x[n]=\sum_{k=-\infty}^{\infty} x[k] \delta[n-k]$. En utilisant la propriété de linéarité, puis d'invariance dans le temps des systèmes LTI, il en vient que

\begin{align}
y[n]&=T_n\left\{x[.]\right\}\\
&=\sum_{k=-\infty}^{\infty} x[k] T_{n}\{\delta[.-k]\}\\
&=\sum_{k=-\infty}^{\infty} x[k] T_{n-k}\{\delta[.]\}\\
&=\sum_{k=-\infty}^{\infty} x[k] h[n-k]
\end{align}
où $h[n]=T_n\{\delta[.]\}$ correspond à la réponse impulsionnelle du filtre.

\section{Démonstration de la proposition \ref{prop_ft}}
\label{dprop_ft}

En appliquant la transformée en $\mathcal{Z}$ de part et d'autre de l'équation (\ref{filtre1}) et en appliquant la propriété de Linéarité de la transformée en $\mathcal{Z}$, nous obtenons
 \begin{align*}
 a_0Y(z)&=\sum_{m=0}^{M}b_m z^{-m} X(z) -\sum_{l=1}^{L}a_l z^{-l}Y(z).
  \end{align*}
 En factorisant par $Y(z)$ et $X(z)$, nous avons
  \begin{align*}
 Y(z)\left(\sum_{l=0}^{L}a_l z^{-l}\right)&=X(z)\left(\sum_{m=0}^{M}b_m z^{-m}\right)
   \end{align*}
 La fonction de transfert $H(z)=Y(z)/X(z)$ s'exprime alors sous la forme
  \begin{align*}
 \frac{Y(z)}{X(z)}&=\frac{\sum_{m=0}^{M}b_m z^{-m}}{\sum_{l=0}^{L}a_l z^{-l}}.
 \end{align*}
 
\section{Démonstration de la proposition \ref{prop1}}
\label{dprop1}
En appliquant la transformée en $\mathcal{Z}$ au filtre (\ref{filtre1}) et en fixant $a_0=1$, nous obtenons 
\begin{align*}
&Y(z)=\sum_{m=0}^{M}b_m X(z)z^{-m}-\sum_{l=1}^{L}a_l Y(z)z^{-l}\\
&Y(z)\left(\sum_{l=0}^{L}a_l z^{-l}\right)=X(z) \left(\sum_{m=0}^{M}b_m z^{-m}\right)
\end{align*}
La fonction de transfert est alors égal à 

\begin{align*}
H(z)\triangleq \frac{Y(z)}{X(z)}=\frac{\sum_{m=0}^{M}b_m z^{-m}}{\sum_{l=0}^{L}a_l z^{-l}}
\end{align*}

\section{Démonstration de la proposition \ref{prop1}}
\label{dprop1}
Considérons la fonction de transfert suivante:
\begin{align*}
H(z)=\left(\frac{b_0}{a_0}\right)\frac{\prod_{k=1}^{M}(1-c_k z^{-1})}{\prod_{l=1}^{N}(1-d_l z^{-1})}
\end{align*}
Sous l'hypothèse que les pôles ne sont pas multiples\footnote{Si les pôles ne sont pas multiples, alors $d_k\ne d_l$ pour tout $k\ne l$.} et que $M<N$, la fonction de transfert peut se décomposer en éléments simples sous la forme
\begin{align}
H(z)=\sum_{l=1}^{N} \frac{r_l}{1-d_l z^{-1}}
\end{align}
où les résidus $r_l$ s'obtiennent via l'expression $r_l=(1-d_l z^{-1})H(z)|_{z=c_l}$. La réponse impulsionnelle $h[n]$ du système s'obtient en inversant la transformée en $\mathcal{Z}$. En utilisant le tableau~\ref{app_tabZ} des transformées en $\mathcal{Z}$, il en vient que 
\begin{align}
h[n]&=\sum_{l=1}^{N} r_l d_l^n u[n]
\end{align}
Le filtre est stable si la réponse impulsionnelle est absolument sommable c-a-d si $\sum_{n=-\infty}^{\infty}|h[n]|<\infty$. Or,
\begin{align}
\sum_{n=-\infty}^{\infty} |h[n]|&\le \sum_{n=0}^{\infty} \sum_{l=1}^{N} \left| r_l \right| \left|d_l \right|^n\\
&=\sum_{l=1}^{N} \left| r_l \right| \sum_{n=0}^{\infty}  \left|d_l \right|^n < \infty
\end{align}
Pour que $\sum_{n=-\infty}^{\infty}|h[n]|<\infty$, il suffit que $|d_l|<1$ pour tout $l$.


\section{Propriété du produit de convolution}
\label{dprop2}
Nous allons montrer que les exponentielles complexes $x[n]=e^{j\omega n}$ sont les invariants du produit de convolution. En utilisant la définition du produit de convolution, nous trouvons
\begin{align}
y[n]&= \sum_{k=-\infty}^{\infty}e^{j\omega k}h[n-k]\\
&= \sum_{m=+\infty}^{-\infty}e^{j\omega (-m+n)}h[m]\\
&= e^{j\omega n}\sum_{m=-\infty}^{\infty}h[m]e^{-j\omega m}\\
&= e^{j\omega n}H(e^{j\omega})
\end{align}


\section{Démonstration de la propriété \ref{prop_f}}
\label{dprop_f}
En utilisant la définition de $X(e^{j\omega})$, nous obtenons

\begin{align*}
\int_{-\pi}^{\pi} X(e^{j\omega})e^{j\omega n}d\omega&=\int_{-\pi}^{\pi} \sum_{m=-\infty}^{\infty}x[m]e^{j\omega (n-m)}d\omega\\
&= \sum_{m=-\infty}^{\infty}x[m]\int_{-\pi}^{\pi}e^{j\omega (n-m)}d\omega\\
&= \sum_{m=-\infty}^{\infty}x[m]I_{n-m}
\end{align*}
où $I_{k}\triangleq \int_{-\pi}^{\pi}e^{jk\omega }d\omega$ et $k\in \mathbb{Z}$. La valeur de l'intégrale $I_{k}$ dépend de la valeur de $k$. Deux cas de figure peuvent se présenter.
\begin{itemize}
\item Si $k=0$, alors $I_{0}=[\omega]_{-\pi}^{\pi}=2\pi$.
\item Si $k\ne 0$, alors 
\begin{align*}
I_{k}&=\frac{1}{jk}[e^{jk\omega }]_{-\pi}^{\pi}\\
&=\frac{1}{jk}(e^{jk\pi}-e^{-jk\pi})\\
&=\frac{2}{k}\sin(k\pi)=0
\end{align*}
\end{itemize}
Il en vient que $I_{k}=2\pi\delta[k]$ et donc
\begin{align*}
\frac{1}{2\pi}\int_{-\pi}^{\pi} X(e^{j\omega})e^{j\omega n}d\omega&=\frac{1}{2\pi} \sum_{m=-\infty}^{\infty}x[m]\delta[n-m]\\
&=\frac{2\pi x[n]}{2\pi}\\
&=x[n].
\end{align*}

\section{Démonstration de la proposition \ref{prop_ii}}
\label{a_prop_ii}
Considérons une fonction de transfert "analogique" ne contenant que des pôles de premier ordre c-a-d
\begin{align}
H_a(s)&=\sum_{k=1}^{N} \frac{A_k}{s-s_k}
\end{align}
En appliquant la transformée de Laplace inverse, nous obtenons la réponse impulsionnelle continue suivante
\begin{align}
h_a(t)&=\left\{
\begin{array}{lc}
\sum_{k=1}^{N} A_k e^{s_k t}&,~t\ge 0,\\
0&,~t<0
\end{array}\right.  
\end{align}
En échantillonnant $h(t)$ avec une période d'échantillonnage $T_e$, nous obtenons
\begin{align}
h[n]&=T_e h_a(t T_e)\\
&=\sum_{k=1}^{N}T_e A_k e^{s_k n T_e}u[n]\\
&=\sum_{k=1}^{N}T_e A_k (e^{s_k T_e})^{n}u[n]
\end{align}
En utilisant le tableau \ref{app_tabZ}, nous obtenons finalement la fonction de transfert en $\mathcal{Z}$ suivante
\begin{align}
H(z)&=\sum_{k=1}^{N} \frac{T_e A_k}{1-e^{s_kT_e}z^{-1}}.
\end{align}

\section{Transformée Bilinéaire}
\label{ap_tf}
La transformée bilinéaire s'obtient en réalisant la substitution
\begin{align}
s&=c\frac{1-z^{-1}}{1+z^{-1}}
\end{align}
Le passage du plan de Laplace vers le domaine en $\mathcal{Z}$ s'obtient alors via l'expression
\begin{align}
z^{-1}&=\frac{1-s/c}{1+s/c}\label{ap_eq_bil_0}
\end{align}
Dans le domaine de Laplace, le domaine fréquentiel s'obtient en posant $s=j\omega_a$. En posant $s=j\omega_a$ dans l'équation (\ref{ap_eq_bil_0}), nous obtenons $|z|=1$. L'axe des imaginaires purs dans le domaine de Laplace est donc transformé en un cercle de rayon unité dans le domaine en $\mathcal{Z}$. De plus, il en vient que 
\begin{align}
j\omega_a&=c\frac{1-e^{-j\omega T_e}}{1+z^{-j\omega T_e}}\\
&=c\frac{e^{j\omega T_e/2}-e^{-j\omega T_e/2}}{e^{j\omega T_e/2}+e^{-j\omega T_e/2}}\\
&=jc\frac{\sin(\omega T_e/2)}{\cos(\omega T_e/2)}\\
&=jc\tan(\omega T_e/2)
\end{align}
La pulsation $\omega_a$ dans le domaine continue est donc déplacé à la pulsation $c\tan(\omega T_e/2)$ dans le domaine numérique. De même, la pulsation numérique $\omega$ correspond à la pulsation $\frac{2}{T_e}\textrm{atan}(\omega_a /c)$ dans le domaine analogique.

\section{Technique de Prony}
\label{ap_prony}

En utilisant l'expression de $h_{df}[n]$, l'erreur tronquée et filtrée $\epsilon_f$ peut se reexprimer sous la forme
\begin{align}
\epsilon_f&=\sum_{n=0}^{N-1}\left(\sum_{l=0}^{L}a_l h_d[n-l]-b[n]\right)^2\\
&=\sum_{n=0}^{N-1}\left[h_d[n]-\left(-\sum_{l=1}^{L}a_l h_d[n-l]+b[n]\right)\right]^2
\end{align}
Comme $b_m=0$ pour tout $k>M$, la somme peut se décomposer en deux parties:
\begin{align}
\epsilon_f&=\sum_{n=0}^{M}\left[h_d[n]-\left(-\sum_{l=1}^{L}a_l h_d[n-l]+b[n]\right)\right]^2\nonumber\\
&+\sum_{n=M+1}^{N-1}\left[h_d[n]-\left(-\sum_{l=1}^{L}a_l h_d[n-l]\right)\right]^2\\
\end{align}
En utilisant les notations matricielles de la propriété \ref{prop_prony}, nous obtenons
\begin{align}
\epsilon_f&=\left\|\textbf{h}+\textbf{H}_0\textbf{a}-\textbf{b}\right \|^2+\left\| \textbf{x}-\textbf{H}\textbf{a}\right \|^2
\end{align}
La première partie est minimisée quelque soit $\textbf{a}$ pour $\widehat{\textbf{b}}=\textbf{h}+\textbf{H}_0\textbf{a}$. En remplacant cette valeur dans l'expression de l'erreur, nous obtenons l'erreur quadratique $\epsilon_f=\left\|\textbf{x}-\textbf{H}\textbf{a}\right\|^2$. Cette erreur est ensuite minimisée en fixant $\widehat{\textbf{a}}=\left(\textbf{H}^T\textbf{H}\right)^{-1}\textbf{H}\textbf{x}$.


\begin{thebibliography}{9}
\bibitem{opp} 
Oppenheim, Alan V and Schafer, Ronald W
\textit{Discrete-time signal processing}.
Pearson Education India, 1999.

\bibitem{smi} 
Smith, Julius Orion
\textit{Introduction to digital filters with audio applications}.
 W3K Publishing, 2008.
 \end{thebibliography}

\end{document}
